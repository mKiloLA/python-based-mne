\section{Course Overview}

DEN 161: Engineering Problem Solving is a lab-style course that complements the lecture-oriented DEN 160: 
Engineering Orientation. The class focuses on providing hands-on, problem solving experiences through projects 
from multiple engineering disciplines. While these projects serve as the students' introduction to different 
engineering disciplines, they also develop the core tools needed to be a successful engineer. 

The current iteration of DEN 161 has three sections of interest to this body of work: data analysis using 
Microsoft Excel, data analysis using Python, and embedded programming using an Arduino Uno.

Microsoft Excel is used to introduce the concept of data analysis to students. Students are tasked with 
manipulating data and finding different statistical properties of given data. This proves to be an effective 
point of entry given to data analysis since many students are familiar with Microsoft Excel or Google Sheets.

Python is then introduced as an alternative method of solving the same problems. The lectures and assignments 
focus on data calculations to verify designs and general code inspection to understand the how the program works. 
These lectures are done using Jupyter Notebook in Anaconda's Spyder IDE. 

Following the introduction to Python, a Stoplight Activity is assigned. This project introduces students to 
circuitry and microcontrollers through the creation of a stoplight using 3 LEDs and an Arduino Uno. The Arduino 
Uno is programmed using C++ in the Arduino IDE.

\section{Project Redesign}

Thanks to the solid programming core created by the instructors, the proposed changes are minor, and result in 
no changes to the current curriculum. Two changes are proposed: the adoption of Visual Studio Code as a 
development environment and the migration from Arduino Unos to Raspberry Pi Picos. 

The class currently uses an IDE by the name of Spyder. Spyder is a popular IDE for data science applications and 
has to ability to seamlessly integrate with Jupyter Notebooks. However, Spyder does not have the ability to work 
with a MicroPython device, such as the RPi Pico. Visual Studio Code, on the other hand, has an extension that 
integrates Pico controls directly into the interface, making it a one-stop-shop for both the data analysis and 
embedded systems development in DEN 161. Visual Studio Code also has Jupyter Notebook extensions that allow for 
a first class experience.

The second proposed change is transitioning from the Arduino Uno to a Raspberry Pi Pico. The reason for this 
change is twofold. First, the Pico can run using MicroPython, a lightweight implementation of Python, which has 
the same syntax as Python. This allows students to focus on understanding one language, Python, rather than 
learning both Python and C++. Switching to the Pico also opens the door to using a single development environment. 
While the Arduino can be programmed using Visual Studio Code, the set up process is non-trivial, and requires a 
strong understanding of the operating and file system of the computer. 

The proposed changes aim to increase student understanding by reducing the number of systems they are introduced 
to. Instead of two langauges and two editors, students will only need to learn one language in one editor. The 
work done by these projects directly correlates with Abet Student Outcomes 6 and 7 and weakly correlates with 
Outcomes 1 and 3, as seen in Appendix \ref{appendix:appendix_abet}.

\section{Project Redesign Assessment}

The project proposal for DEN 161 is unique from the other classes in this work because the proposed changes were
implemented into the standard class curriculum. The second week of Python was replaced with the tire RPM problem
and the Arduino Uno was replaced by the Raspberry Pi Pico in the stoplight project. The adoption was largely 
successful, and the instructors felt that it was a step in the correct direction. That is not to say that there
were no pain points.

The biggest hinderance encountered in the Fall 2023 was general file system comprehension. Many students did not 
know where downloaded files go or how the file system (both File Explorer and Finder) were structured. This proved 
to be an issue for both the tire RPM question and the stoplight example. For Python to locate the .csv file 
with the RPM data, either the full path to the file needs to be provided, relative to the current working directory,
or the file needed to be in the same folder as the script. To get the MicroPico extension to work correctly in
Visual Studio Code, a working directory needs to be set. 

Both of these issues can likely be solved with the same two methods. First, time will be spent in a previous lecture
to talk about the file system. This is basic information that is required to effectively use a computer. Second,
files for assignments will be distributed in zip files. This will nearly guarantee that files have the correct
relative location. This will introduce a new issue of trying to link and edit files in a zip folder, but hopefully
it is an easier fix than trying to hunt down files on student's laptops.

The Fall 2023 semester also found that using Visual Studio Code for both the data analysis and Pico portions of the
class is better for students than using Spyder for the data analysis and VSCode for the Pico. 

An additional pre-class assignment will also be created that will require students to verify that they have
done the necessary installations before coming to class. Though this is not an issue directly caused by the changes
made, it did directly inhibit student learning.

Lastly, and most concerningly, this semester showed that the MicroPico extension can be unreliable when not used
exactly as intended. While no connection issues ever occurred during testing, a non-trivial number of students
experienced issues with the MicroPico extension. Since no issues have been encountered by the instructors at this
point, it is hard to pinpoint the exact cause. Fixes for different issues can be found in the 
"usage-and-installation" folder in the GitHub Repository.

\section{Project Deliverables}

In the GitHub repository associated with this paper, which can be found in Appendix \ref{appendix:appendix_github},
the folder titled ``engineering-problem-solving,'' there are several folders containing different problems and 
examples. The folder also contains a README that details what is in each file and what software is needed to 
complete the assignments. 

The first folder is a tire RPM example problem that is to be completed in class, after having completed the Excel
lectures and the introduction to Python lectures. This will show students some of the benefits of using a
language like Python over Excel. To go along with the example problem is a similar tire RPM homework assignment.
The code needed to complete the assignment is all provided to students, with a handful of tasks left up to them,
as detailed in the homework file. 

Finally, the stoplight folder contains the starter code to give students for the stoplight assignment, a solution
key, and a file demonstrating morse code on the Pico. Installation instructions for the using the Pico can be found
in the ``usage-and-installation'' folder in the repository.
