\section{Course Overview}

DEN 161: Engineering Problem Solving is a lab-style course that complements the lecture-oriented DEN 160: 
Engineering Orientation. The class focuses on providing hands-on, problem-solving experiences through projects 
from multiple engineering disciplines. While these projects serve as the students' introduction to different 
engineering disciplines, they also develop the core tools needed to be a successful engineer. 

The current iteration of DEN 161 has three sections of interest to this body of work: data analysis using 
Microsoft Excel, data analysis using Python, and embedded programming using an Arduino Uno.

Microsoft Excel is used to introduce the concept of data analysis to students. Students are tasked with 
manipulating data and finding different statistical properties of given data. This proves to be an effective 
point of entry given to data analysis since many students are familiar with Microsoft Excel or Google Sheets.

Python is then introduced as an alternative method of solving the same problems. The lectures and assignments 
focus on data calculations to verify designs and general code inspection to understand how the program works. 
These lectures are done using Jupyter Notebook in Anaconda's Spyder IDE. 

Following the introduction to Python, a Stoplight Activity is assigned. This project introduces students to 
circuitry and microcontrollers through the creation of a stoplight using 3 LEDs and an Arduino Uno. The Arduino 
Uno is programmed using C++ in the Arduino IDE.

\section{Project Redesign}

Thanks to the solid programming core created by the instructors, the proposed changes are minor, and result in 
few changes to the current curriculum. Two changes are proposed: the adoption of Visual Studio Code as a 
development environment and the migration from Arduino Unos to Raspberry Pi Picos. 

The class currently uses an IDE by the name of Spyder. Spyder is a popular IDE for data science applications and 
has the to ability to seamlessly integrate with Jupyter Notebooks. However, Spyder does not have the ability to work 
with a MicroPython device, such as the RPi Pico. Visual Studio Code, on the other hand, has an extension that 
integrates Pico controls directly into the interface, making it a one-stop-shop for both the data analysis and 
embedded systems development in DEN 161. Visual Studio Code also has Jupyter Notebook extensions that allow for 
a first class experience.

The second proposed change is transitioning from the Arduino Uno to a Raspberry Pi Pico. The reason for this 
change is twofold. First, the Pico can run using MicroPython, a lightweight implementation of Python, which has 
the same syntax as Python. This allows students to focus on understanding one language, Python, rather than 
learning both Python and C++. Switching to the Pico also opens the door to using a single development environment. 
While the Arduino can be programmed using Visual Studio Code, the setup process is non-trivial, and requires a 
strong understanding of the operating and file system of the computer. 

With these changes in mind, the first two assignments created serve as a segue from using Excel for data analysis to 
using Python. The first of the two, Assignment \ref{engg_prob_solve_assignment_1}, requires students to convert 
tire rpm data to speed data, find several statistical properties of the data, and then plot it. The assignment
is done in class both in Excel and Python to showcase the differences.

\assignments{DEN 161: Assignment 1}
\label{engg_prob_solve_assignment_1}

\begin{tcolorbox}[breakable, enhanced jigsaw, title=DEN 161: Assignment \ref{engg_prob_solve_assignment_1}, 
    colframe=ksu-purple, colback=ksu-gray]

    \textbf{Problem Statement}
    \parindent15pt

    You are given data for the tire rotations per minute in a file called `tire\_rpm.csv' and are tasked with 
    finding the max, min, mean, mode, and median speed of the vehicle. Bonus points for a graph!

    \tcblower
    \textbf{Problem Solution}
    \parindent15pt

    For the full solution, see Appendix \ref{appendix:appendix_github}. The following shows the process
    of reading information from a .csv file.
    
    To solve this problem, let's split it into steps and address the issues one at a time.

    \begin{enumerate}
        \item Read and parse the data from the .csv file
        \item Convert the RPM data into vehicle speed 
        \item Find the max, min, mean, mode, and median
    \end{enumerate}
    
    \noindent \textbf{Step 1: Read and parse the data}
    
    To begin, we will import the built-in csv library. This Python package makes it easy to parse data from csv 
    files, and takes away the burden of writing code to split the data. We will also import the math library.

\begin{python}
import csv
import math
\end{python}

    We will start by creating an empty list for our rpm values to be stored in. Next we will open the .csv 
    file and read it.

\begin{python}
with open("tire_rpm_example.csv", "r") as file_contents:
csv_reader = csv.reader(file_contents,delimiter=",")
rpm = []
for rpm_row in csv_reader:
    for rpm_value in rpm_row:
        rpm.append(int(rpm_value))
print(f"Tire RPM: {rpm}")
\end{python}
\end{tcolorbox}

After completing the in-class example, a similar assignment is given for homework, as shown in Assignment
\ref{engg_prob_solve_assignment_2}. Rather than writing the code from scratch, students must make several 
changes and add comments to the file.

\assignments{DEN 161: Assignment 2}
\label{engg_prob_solve_assignment_2}

\begin{tcolorbox}[breakable, enhanced jigsaw, title=DEN 161: Assignment \ref{engg_prob_solve_assignment_2}, 
    colframe=ksu-purple, colback=ksu-gray]

    \textbf{Problem Statement}
    \parindent15pt

    Make the following changes to the given Python file:

    \begin{enumerate}
        \item Change the input file to the homework data set.
        \item Change the tire diameter to utilize user input.
        \item Change the print statements to have 0 decimal places.
        \item Anywhere there is a \#COMMENT, leave a comment explaining what the following lines of code do.
    \end{enumerate}
               
    Optionally, complete the bonus assignment question at the bottom of the file.
    

    \tcblower
    \textbf{Problem Solution}
    \parindent15pt

    For the full solution, see Appendix \ref{appendix:appendix_github}. The following shows the changed file
    name, user input, and descriptive comments.

\begin{python}
# Import statements to make library methods 
# available in this file
import csv
import math
import statistics
from matplotlib import pyplot

# Open and read data from a csv file into Python
with open("tire_rpm_homework.csv","r", 
          encoding="utf-8") as file_contents:
    csv_reader = csv.reader(file_contents,delimiter=",")
    rpm = []
    for rpm_value in next(csv_reader):
        # Turn the rpm_value (which would be a string) 
        # into an integer and add it to a list of rpms
        rpm.append(int(rpm_value))

# Ask the user to input the tire diameter
tire_diameter = input("What is the tire diameter?")
\end{python}
\end{tcolorbox}

The third assignment, shown in Assignment \ref{engg_prob_solve_assignment_3} was translated from the 
current Stoplight project used in the class \cite{stoplight}. An example file
demonstrating how to operate the LEDs serves as the starting point for the project. To complete the
assignment, students need to rearrange the \pyth{pin.high()} calls and change the sleep timings.

\assignments{DEN 161: Assignment 3}
\label{engg_prob_solve_assignment_3}

\begin{tcolorbox}[breakable, enhanced jigsaw, title=DEN 161: Assignment \ref{engg_prob_solve_assignment_3}, 
    colframe=ksu-purple, colback=ksu-gray]

    \textbf{Problem Statement}
    \parindent15pt

    Using the Raspberry Pi Pico and three LEDs given out in class, write a program that will operate
    the LEDs like a stoplight. For bonus assignment credit, write a program that will blink SOS in Morse code.
    
    \tcblower
    \textbf{Problem Solution}
    \parindent15pt
    
    For the full bonus assignment solution and the example file, see Appendix \ref{appendix:appendix_github}. 
    The following shows the entire solution to the stoplight portion of the assignment.

\begin{python}
from machine import Pin
import time

# Declare the pins and the pin mode
red_led = Pin(18, Pin.OUT)
yellow_led = Pin(17, Pin.OUT)
green_led = Pin(16, Pin.OUT)

# Begin looping phase
while True:
    # Turn on the red led for 5 seconds
    green_led.low()
    yellow_led.low()
    red_led.high()
    time.sleep(5)

    # Turn green led on and red led off for 5 seconds
    green_led.high()
    red_led.low()
    time.sleep(5)

    # Turn yellow led on and green led off for 2 seconds
    yellow_led.high()
    green_led.low()
    time.sleep(2)
    \end{python}
\end{tcolorbox}

The proposed changes aim to increase student understanding by reducing the number of systems they are introduced 
to. Instead of two languages and two editors, students will only need to learn one language in one editor. The 
work done by these projects directly correlates with ABET Student Outcomes 6 and 7 and weakly correlates with 
Outcomes 1 and 3, as seen in Appendix~\ref{appendix:appendix_abet}.

\section{Project Redesign Assessment}

The project proposal for DEN 161 is unique from the other classes in this work because the proposed changes were
implemented into the standard class curriculum. The second week of Python was replaced with the tire RPM problem
and the Arduino Uno was replaced by the Raspberry Pi Pico in the stoplight project. The adoption was largely 
successful, and the instructors felt that it was a step in the correct direction. That is not to say that there
were no pain points.

The biggest hindrance encountered in the Fall 2023 was general file system comprehension. Many students did not 
know where downloaded files go or how the file system (both File Explorer and Finder) were structured. This proved 
to be an issue for both the tire RPM question and the stoplight example. For Python to locate the .csv file 
with the RPM data, either the full path to the file needs to be provided, relative to the current working directory,
or the file needed to be in the same folder as the script. To get the MicroPico extension to work correctly in
Visual Studio Code, a working directory needs to be set. 

Both of these issues can likely be solved with the same two methods. First, time will be spent in a previous lecture
to talk about the file system. This is basic information that is required to effectively use a computer. Second,
files for assignments will be distributed in zip files. This will nearly guarantee that files have the correct
relative location. This will introduce a new issue of trying to link and edit files in a zip folder, but hopefully
it is an easier fix than trying to hunt down files on student's laptops.

The Fall 2023 semester also found that using Visual Studio Code for both the data analysis and Pico portions of the
class is better for students than using Spyder for the data analysis and VSCode for the Pico. 

An additional pre-class assignment will also be created that will require students to verify that they have
done the necessary installations before coming to class. Though this is not an issue directly caused by the changes
made, it did directly inhibit student learning.

The final issue, and perhaps the most concerning, is the unreliability of the MicroPico extension when not used 
exactly as intended. While no connection issues ever occurred during testing, a non-trivial number of students
experienced issues with the MicroPico extension. Since no issues have been encountered by the instructors at this
point, it is hard to pinpoint the exact cause. Fixes for different issues can be found in the 
``usage-and-installation'' folder in the GitHub Repository.

\section{Recommendations for Course Integration}

After completing the new (and translated) assignments in DEN 161, students should be able to identify the components 
used in simple programs and explain the functionality of code blocks. With this understanding, students should also 
be able to use simple programming techniques to alter the functionality of existing code. 

To achieve these outcomes, DEN 161 should continue teaching its introduction to Python as well as continue
using the adopted material presented above. In response to the first semester assessment, Visual Studio
Code should be adopted over Spyder for the Python assignments, and files should be distributed through
the use of zip files to prevent issues with file location. To contend with the issues caused by the lack
of file system knowledge, as well as to preemptively address the issues with extracting zip files, additional
class time should be spent ensuring students understand the basics of a file system. This ability to use
a file system extends far past programming and is essential for every student.

\section{Project Deliverables}

In the GitHub repository associated with this paper, which can be found in Appendix \ref{appendix:appendix_github},
the folder titled ``3-engineering-problem-solving,'' there are several folders containing different problems and 
examples. The folder also contains a README that details what is in each file and what software is needed to 
complete the assignments. 

The first folder is a tire RPM example problem that is to be completed in class, after having completed the Excel
lectures and the introduction to Python lectures. This will show students the benefits of using a
language like Python over Excel. To go along with the example problem is a similar tire RPM homework assignment.
The code needed to complete the assignment is all provided to students, with a handful of tasks left up to them,
as detailed in the homework file. 

Finally, the stoplight folder contains the starter code to give students for the stoplight assignment, a solution
key, and a file demonstrating Morse code on the Pico. Installation instructions for the using the Pico can be found
in the ``usage-and-installation'' folder in the repository.
