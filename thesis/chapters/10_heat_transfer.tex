\section{Course Overview}

ME 573: Heat Transfer is lecture based class that focuses on analyzing systems using
the three modes of heat transfer: conduction, convection, and radiation. The study of
heat transfer is dense with equations, analysis methods, and tables, most of which
provide opportunities for programming to be used for solving problems, whether that
be Excel calculators or Python. Of these, perhaps the most practical application 
is the design of heat exchangers. Accordingly, the only project in the class is an 
iterative heat exchanger design problem, which requires the use of Excel or Python.

\section{Project Modifications}

Since a project that makes use of programming already exists for Heat Transfer, no
changes have been made to the assignment. The problem statement comes from 
\textit{Fundamentals of Heat and Mass Transfer, 7th Edition} \cite{heat-transfer} and 
can be found unchanged in the 10-heat-transfer repository along with a solution made 
using Python in a Jupyter Notebook.

The project assignment, shown in Assignment \ref{heat_transfer_assignment_1}, deals 
with the design of a shell and tube heat exchanger. The solution to the problem requires
table data, iteration, and plotting, making it a prime example of the practical benefits
of programming in the field. A library that fully implements the IF-97 standard called
$pyXSteam$ is used for steam and water properties \cite{pyxsteam2020}.

\assignments{ME 573: Assignment 1}
\label{heat_transfer_assignment_1}

\begin{tcolorbox}[breakable, enhanced jigsaw, title=ME 573: Assignment \ref{heat_transfer_assignment_1}, 
    colframe=ksu-purple, colback=ksu-gray]

    \textbf{Problem Statement}
    \parindent15pt

    A heat exchanger manufacturing firm has hired your team to design a shell and tube 
    heat exchanger. The heat exchanger is expected to heat 10,000 $\frac{kg}{h}$ of water flowing 
    through the tubes from 16 to 84 °C. The hot engine oil flowing through the shell is
    used for heating the water. The oil makes a single shell pass, entering at 160 °C and 
    leaving at 94 °C, with an average overall heat transfer coefficient $U_h = 400 \frac{W}{m^2K}$. The water
    flows through 11 brass tubes (k = 137 $\frac{W}{m K}$) of 22.9 mm inside diameter and 25.4 mm outside
    diameter, with each tube making four passes through the shell, as shown below.
    
    Declare known variables, make necessary imports, and interpolate for property values.

    \tcblower
    \textbf{Problem Solution}
    \parindent15pt

    For the full solution, see Appendix \ref{appendix:appendix_github}. The following code shows
    how to access and use the steam tables using $pyXSteam$.

\begin{python}
from pyXSteam.XSteam import XSteam

# m/kg/sec/K/MPa/W
steam_table = XSteam(XSteam.UNIT_SYSTEM_BARE)

v = steam_table.vL_t(323) # m^3/kg
Cpc = 1000*steam_table.CpL_t(323) # J/(kg*K)
mew = steam_table.my_pt(0.1, 323) # N*s/m^2
kwater = steam_table.tcL_t(323) # W/(m*K)
\end{python}

Unfortunately, no Python library with engine oil data currently exists, so values 
will have to come from a textbook. The following function was made using $scipy.optimize.curve\_fit$ 
and a fourth order polynomial with the data from Appendix A: Table A.5 from \textit{Fundamentals of Heat 
and Mass Transfer} by Bergman, Lavine, Incropera, and Dewitt \cite{heat-transfer}.

\begin{python}
def oil_specific_heat(x):
    a,b,c,d,e = 7.57640461e-07, -1.16948468e-03,  6.76308996e-01, -1.69259798e+02, 1.72835141e+04
    return (a*x**4+b*x**3+c*x**2+d*x+e)

Cph = oil_specific_heat(400) # J/(kg*K)
\end{python}
\end{tcolorbox}

Since no new assignments or requirements are being added, the learning objectives
for the class do not change. 

\section{Recommendations for Course Integration}

After completing ME 573, students should be able to organize code in a structured manner, 
argue about the functionality of written code, and produce high quality code with no prompting
or examples.

To accomplish this, ME 573 should continue assigning a heat exchanger design project that 
requires the use of programming to complete. This serves as a suitable wrapper to the skills
taught in previous classes by requiring additional libraries, graphing, iteration, and design.
Optionally, adding smaller programming questions onto regular assignments would help solidify
student understanding. 

Little should be done by the instructor to teach students how to program or how to use any
particular libraries, though a general suggestion of which libraries to use would be appropriate.
This is to encourage students to apply the skills they have learned up to this point rather than
providing an example to copy from. 

\section{Project Deliverables}

In the GitHub repository associated with this paper, which can be found in 
Appendix \ref{appendix:appendix_github}, the folder titled ``10-heat-trasnfer''
contains both the problem statement and solution guide for the project introduced in 
the previous section. The folder also contains a README that details what is in each file and 
what software is needed to complete the assignments. 

For these projects to be added to the class, the instructor would simply need to give the 
starting code to students as a problem statement. Giving an example of how to set the units
and pull data from the $pyXSteam$ library would be a useful addition to the in-class presentation
of the project.
