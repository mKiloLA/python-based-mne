\section{Course Overview}

ME 573: Heat Transfer is lecture based class that focuses on analyzing systems using
the three modes of heat transfer: conduction, convection, and radiation. The study of
heat transfer is dense with equations, analysis methods, and tables, most of which
provide opportunities for programming to be used for solving problems, whether that
be Excel calculators or Python. Of these, perhaps the most practical application 
is the design of heat exchangers. Accordingly, the only project in the class is an 
iterative heat exchanger design problem, which requires the use of Excel or Python.

\section{Project Modifications}

Since a project that makes use of programming already exists for Heat Transfer, no
changes have been made to the assignment. The problem statement can be found unchanged
in the heat-transfer repository as well as a solution made using Python in a Jupyter
Notebook.

In lieu of changing the project itself, additional resource were made to mimic the 
functionality of an MNE Python library. The library contains methods for getting 
the properties of water and oil needed to complete the project. Rather than saving
large tables of data, fourth order polynomials were fit to the data using 
`scipy.optimize.curve\_fit' to allow for a smaller package size. The purpose-built
library also has a function for the Reynolds number. 

While this library currently contains exactly what is needed to complete the project,
the end-of-the-line goal is a library that contains every property and equation that
could be needed.

Since no new assignments or requirements are being added, the learning objectives
for the class do not change. 

\section{Project Deliverables}

In the GitHub repository associated with this paper, which can be found in 
Appendix \ref{appendix:appendix_github}, the folder titled ``10-heat-trasnfer''
contains both the problem statement and solution guide for the project introduced in 
the previous section. The folder also contains a README that details what is in each file and 
what software is needed to complete the assignments. 

For these projects to be added to the class, the instructor would simply need to give the 
skeleton files to students as a problem statement. It may be beneficial to give students 
an example of how to use the look-up methods in the MNE library. 
