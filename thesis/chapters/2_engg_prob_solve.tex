\section{Course Overview}
Dean of Engineering 161: Engineering Problem Solving is a lab-style
course that complements the lecture-oriented DEN 160: Engineering
Orientation. The class focuses on providing hands-on, problem
solving experiences through projects from multiple engineering 
disciplines. While these projects serve as the students' introduction
to different engineering disciplines, they also develop the core 
tools needed to be a successful engineer. 

The current iteration of DEN 161 has three sections of interest to
this body of work: data analysis using Microsoft Excel, data analysis
using Python, and embedded programming using an Arduino Uno.

Microsoft Excel is used to introduce the concept of data analysis
to students. Students are tasked with manipulating data and 
finding different statistical properties of given data. 
This proves to be an effective point of entry given to data
analysis since many students are familiar with Microsoft Excel or 
Google Sheets.

Python is then introduced as an alternative method of solving the same
problems. The lectures and assignments focus on data calculations to 
verify designs and general code inspection to understand the 
how the program works. These lectures are done using Jupyter
Notebook in Anaconda's Spyder IDE. 

Following the introduction to Python, a Stoplight Activity is assigned. 
This project introduces students to circuitry and microcontrollers 
through the creation of a stoplight using 3 LEDs and an Arduino Uno.
The Arduino Uno is programmed using C++ in the Arduino IDE.

\section{Proposed Changes}
Thanks to the solid programming core created by the instructors,
the proposed changes are minor, and result in no changes to the
current curriculum. Two changes are proposed: the adoption of Visual 
Studio Code as a development environment and the migration from 
Arduino Unos to Raspberry Pi Picos. 

The class currently uses an IDE by the name of Spyder. Spyder is a
popular IDE for data science applications and has to ability to 
seamlessly integrate with Jupyter Notebooks. However, Spyder does
not have the ability to work with a MicroPython device, such as the
RPi Pico. Visual Studio Code, on the other hand, has an extension 
that integrates Pico controls directly into the interface, 
making it a one-stop-shop for both the data analysis and embedded 
systems development in DEN 161. Visual Studio Code also has 
Jupyter Notebooks extensions that allow for a first class experience.

The second proposed change is transitioning from the Arduino Uno
to a Raspberry Pi Pico. The reason for this change is twofold. First,
the Pico can run using MicroPython, a lightweight implementation of
Python, which has the same syntax as Python. This allows students
to focus on understanding one language, Python, rather than learning
both Python and C++. Switching to the Pico also opens the door to
using a single development environment. While the Arduino can
be programmed using Visual Studio Code, the set up process
is non-trivial, and requires a strong understanding of the
operating and file system of the computer. 

The proposed changes aim to increase student understanding
by reducing the number of systems they are introduced to. Instead
of two langauges and two editors, students will only need to
learn one language in one editor. The work done by these projects
directly correlates with student outcomes 6 and 7 and weakly
correlates with outcomes 1 and 3 as seen in Appendix 
\ref{appendix:appendix_abet}.

\section{Project Deliverables}
The repository contains in-class examples, assignments, and 
solutions for data analysis in Excel, data analysis in Python, 
and programming the Raspberry Pi Pico. It also contains an 
installation guide, source code for the custom extension pack, 
and videos walking through the setup and completition of the 
different assignments. Included below is an instructors guide
that walks through the assigments and files in the repository.

\subsection{Instructors Guide}
Change the order to start with basic Python, which teachers
already have, then the excel problem and the python version of
the csv file. Then move to using the pico and ciruit building

\begin{enumerate}
    \item \textbf{Introduction to Excel}
        \begin{itemize}
            \item The content for the introduction to Excel
            already exists and does not need to be updated.
        \end{itemize}
    \item \textbf{Introduction to Python}
        \begin{itemize}
            \item The content for the introduction to Python
            already exists and does not need to be updated.
        \end{itemize}
    \item \textbf{Data Analysis with Excel}
        \begin{itemize}
            \item \textbf{In-Class Example:} Using the data in 
            tire\_rpm\_excel.csv, find the maximum, minimum, 
            mean, median, and mode speed of the vehicle. Assume 
            a tire diameter of 20 inches. Graph the speed of 
            the vehicle at any given time.
            \item \textbf{Homework:} Have students repeat the 
            process using the data from tire\_rpm\_homework.csv 
            and 18" wheels. Graph the speed of the vehicle at 
            any given time. This should be a 1-to-1 copy of what 
            was done in class, just with different numbers.
        \end{itemize}
    \item \textbf{Data Analysis with Python}
        \begin{itemize}
            \item \textbf{In-Class Example:} Using the data from 
            tire\_rpm\_example.csv and the Jupyter Notebook 
            intro\_to\_python\_example.ipynb, find the maximum, 
            minimum, mean, median, and mode speed of the vehicle. 
            Assume a tire diameter of 20 inches. Graph the speed 
            of the vehicle at any given time. This should give 
            identical results to the Excel example problem.
            \item \textbf{Homework:} Have students create a .py 
            file that finds the maximum, minimum, mean, median, 
            and mode speed of the vehicle using 
            tire\_rpm\_homework.csv. Assume a tire diameter of 
            18 inches. Graph the speed of the vehicle at any 
            given time.
                \begin{itemize}
                    \item \textbf{Extra Credit:} How long did 
                    it take a car with 22" wheels to go 0-60 
                    if the sensor data was taken at 300Hz?
                \end{itemize}
        \end{itemize}
    \item \textbf{Programming the Raspberry Pi Pico}
        \begin{itemize}
            \item \textbf{In-Class Example:} Walk through the code 
            in example.py to show students how to blink the LEDs. 
            \item \textbf{Homework:} Task students with altering 
            the code provided in class to make the LEDs function 
            as a stoplight. A potential solution is provided in 
            stoplight.py.
            \begin{itemize}
                \item \textbf{Extra Credit:} Make the LEDs spell 
                your name in morse code. An example of looping morse
                code is shown in sos.py.
            \end{itemize}
        \end{itemize}
\end{enumerate}

\subsection{Description of Files in the Repository}
See Appendix \ref{appendix:appendix_github} for full source code
and documentation. Videos have been removed from the repository
due to file size limitations.
\begin{enumerate}
    \item \textbf{installation\_guides}
        \begin{itemize}
            \item \textbf{Installation\_Guide.pdf:} a guide 
            that walks through downloading Anaconda, Visual 
            Studio Code, and the KSU Extensions in Visual 
            Studio Code.
        \end{itemize}
    \item \textbf{intro\_to\_excel}
        \begin{itemize}
            \item \textbf{tire\_rpm\_example.csv:} a data file 
            that contains RPM data for a car wheel. Use this 
            data for the example questions.
            \item \textbf{tire\_rpm\_homework.csv:} a data file 
            that contains RPM data for a car wheel. Use this 
            data for the homework questions.
            \item \textbf{tire\_rpm\_example\_solution.xlsx:} a 
            potential solution to the in-class problem posed in 
            Step 1.
            \item \textbf{tire\_rpm\_homework\_solution.xlsx:} a 
            potential solution to the homework problem posed in 
            Step 1.
        \end{itemize}
    \item \textbf{intro\_to\_python}
        \begin{itemize}
            \item \textbf{tire\_rpm\_example.csv:} a data file 
            that contains RPM data for a car wheel. Use this 
            data for the example questions.
            \item \textbf{tire\_rpm\_homework.csv:} a data file 
            that contains RPM data for a car wheel. Use this 
            data for the homework questions.
            \item \textbf{intro\_to\_python\_example.ipynb:} a 
            Jupyter Notebook file that walks through solving the 
            in-class example problem. This file is intended to 
            bridge the gap between Excel and Python.
            \item \textbf{intro\_to\_python\_homework\_solution.py:} 
            a Python script for solving the homework question from 
            Step 2. This could also be done in a Jupyter Notebook, 
            but using a .py file was used to showcase standard
            Python usage. This also contains the solution to the 
            extra credit question.
        \end{itemize}
    \item \textbf{intro\_to\_pico}
        \begin{itemize}
            \item \textbf{example.py:} LED flashing program for the 
            RPi Pico. This file is intended to be used as the in-class 
            example in Step 3 and the base code provided for the homework.
            \item \textbf{stoplight.py:} this is a potential solution 
            to the Stoplight Activity. Many different variations of 
            this file could exist.
            \item \textbf{sos.py:} this is an example of using 
            morse code with the Pico. The solution utilizes looping
            and a function to reduce repeated code.
        \end{itemize}
    \item \textbf{ksu\_den\_161\_extension\_pack}
        \begin{itemize}
            \item \textbf{package.json:} this file contains the code 
            used to create the Extension Package in the Microsoft 
            Marketplace. As it stands, this file (and folder) can be 
            ignored. In the future, an instructor will need to make 
            sure the extension pack stays up to date.
        \end{itemize}
\end{enumerate}