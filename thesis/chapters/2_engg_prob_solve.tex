\section{Course Overview}
Dean of Engineering 161: Engineering Problem Solving is a lab-style
course that complements the lecture-oriented DEN 160: Engineering
Orientation. The class focuses on providing hands-on, problem
solving experiences through projects from multiple engineering 
disciplines. While these projects serve as the students' introduction
to different engineering disciplines, they also develop the core 
tools needed to be a successful engineer. 

The current iteration of DEN 161 has three sections of interest to
this body of work: data analysis using Microsoft Excel, data analysis
using Python, and embedded programming using an Arduino Uno.

Microsoft Excel is used to introduce the concept of data analysis
to students. Students are tasked with manipulating data and 
finding different statistical properties of given data. 
This proves to be an effective point of entry given to data
analysis since many students are familiar with Microsoft Excel or 
Google Sheets.

Python is then introduced as an alternative method of solving the same
problems. The lectures and assignments focus on data calculations to 
verify designs and general code inspection to understand the 
how the program works. These lectures are done using Jupyter
Notebook in Anaconda's Spyder IDE. 

Following the introduction to Python, a Stoplight Activity is assigned. 
This project introduces students to circuitry and microcontrollers 
through the creation of a stoplight using 3 LEDs and an Arduino Uno.
The Arduino Uno is programmed using C++ in the Arduino IDE.

\section{Proposed Changes}
Thanks to the solid programming core created by the instructors,
the proposed changes are minor, and result in no changes to the
current curriculum. Two changes are proposed: the adoption of Visual 
Studio Code as a development environment and the migration from 
Arduino Unos to Raspberry Pi Picos. 

The class currently uses an IDE by the name of Spyder. Spyder is a
popular IDE for data science applications and has to ability to 
seamlessly integrate with Jupyter Notebooks. However, Spyder does
not have the ability to work with a MicroPython device, such as the
RPi Pico. Visual Studio Code, on the other hand, has an extension 
that integrates Pico controls directly into the interface, 
making it a one-stop-shop for both the data analysis and embedded 
systems development in DEN 161. Visual Studio Code also has 
Jupyter Notebooks extensions that allow for a first class experience.

The second proposed change is transitioning from the Arduino Uno
to a Raspberry Pi Pico. The reason for this change is twofold. First,
the Pico can run using MicroPython, a lightweight implementation of
Python, which has the same syntax as Python. This allows students
to focus on understanding one language, Python, rather than learning
both Python and C++. Switching to the Pico also opens the door to
using a single development environment. While the Arduino can
be programmed using Visual Studio Code, the set up process
is non-trivial, and requires a strong understanding of the
operating and file system of the computer. 

The proposed changes aim to increase student understanding
by reducing the number of systems they are introduced to. Instead
of two langauges and two editors, students will only need to
learn one language in one editor. The work done by these projects
directly correlates with student outcomes 6 and 7 and weakly
correlates with outcomes 1 and 3 as seen in Appendix 
\ref{appendix:appendix_abet}.

\section{Project Deliverables}
