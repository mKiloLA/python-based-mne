\section{Course Overview}
Dean of Engineering 161: Engineering Problem Solving is a lab-style
course that complements the lecture-oriented DEN 160: Engineering
Orientation. The class focuses on providing hands-on, problem
solving experiences through projects from multiple engineering 
disciplines. While these projects serve as the students' introduction
to different engineering disciplines, they also develop the core 
tools needed to be a successful engineer. 

\section{Project Integration}
Following a two week introduction to Microsoft Excel, the current 
iteration of DEN 161 spends two weeks learning the basics of Python. 
The Python lectures and assignments focus on data calculations to 
verify designs and general code inspection to understand the 
different available options. These lectures are done using Jupyter
Notebook in Anaconda's Spyder IDE. 

After Python, the class begins working on the Stoplight Activity.
This project introduces students to circuitry and microcontrollers 
through the creation of a stoplight using 3 LEDs and an Arduino Uno.


After lecturing on Python,
two more weeks are spent doing a Stoplight Activity that 
utilizes an Arduino Nano, the Arduino IDE, and C++. The primary goal 
of the project is to give students a baseline understanding of what 
programming and electronics make possible.



To build upon the two weeks of Python, as well as maintaining a 
consistent development environment, the stoplight activity will be 
redesign to utilize a Raspberry Pi Pico as the primary microcontroller. The IDE will be Visual Studio Code, which requires an extension to easily work on the Pico. The Pico will be flashed with MicroPython, a light-weight variant of Python.

\section{Project Description and Deliverables}
Description of new assignment?

Pictures of circuit?

Code blocks?

Lab Assignment and solution Guide?

I wish it would listen to my changes