Talk about what classes currently utilize programming and electronics 
and what languages/hardware they use. Discuss experience as instructor 
of controls/design 1/2 where students do not know how to code at 
all - even simple scripting (this makes it hard to teach controls or 
design when students do not even know how to operate the platform they
are learning on). Also talk about the main components that will be 
used to complete the projects: pico, vscode, python

\section{Motivation}

The two weeks of Python in Spyder will be switched to using Visual Studio Code. This change is made for three primary reasons:

* Visual Studio Code is the industry standard for code developement. This means it has the highest level of support for both extensions and bug fixes.
* Anaconda is a package of Python that includes many data-analytics and machine learning focused packages. While this can be convenient, it adds a layer of confusion to new users regarding the difference between Python/Anaconda/Spyder that could be avoided.
* Visual Studio Code has an extension made for microPython. While an more user-friendly IDE for the Raspberry Pi Pico does exist, the ability to use one environment for both Python and microPython makes for a more seamless experience.

\section{Introduction to Python}

\section{Introduction to the Raspberry Pi Pico}

\section{Introduction to Visual Studio Code}