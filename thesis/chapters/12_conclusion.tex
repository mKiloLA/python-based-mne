\section{Future Work}

While this project shows the possibility of uniting programming in the mechanical 
engineering curriculum, several additional areas need to be explored before a 
wide-spread adoption. 

The first is a study to substantiate the assumed claim that a more 
consistent approach to programming would improve students' abilities to solve
engineering problems. Additionally, alternative programming languages should be
reviewed, such as the use of C++, MATLAB, Octave, or Scilab. 
Finally, continued work on the creation, documentation, and distribution
of an MNE library that contains 3rd party libraries, equations, and tables
will solidify the cohesive nature of the programming curriculum and create an
easier method for version controlling the software downloaded by students.

\section{Conclusion}

The work in this project demonstrates that a cohesive and consistent programming 
ecosystem could be adopted throughout the curriculum of the Mechanical and 
Nuclear Engineering department at Kansas State. Whether by an adapted or translated
assignment, each relevant class was able to utilize Python as a problem-solving 
tool. 

This work showcases both the simplicity of setup and ease of use for students
while demonstrating the capabilities and convenience of Python thanks to the 
emphasis on code readability and plethora of engineering related libraries.
In cases of translation, Python was able to closely imitate the current language,
both in for embedded programming and controls, while maintaining the same 
working environment and syntax. By introducing a unified programming curriculum
to the core Kansas State Mechanical Engineering classes, students will be better
equipped to solve real engineering problems.
