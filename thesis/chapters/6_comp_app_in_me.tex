\section{Course Overview}

ME 400: Computer Applications in Mechanical Engineering is a lecture based class with
one lab per week. The lectures initially emphasis learning the basic tools of
programming, looping, functions, classes, etc., while the remainder of the class
focuses on embedded programming and microcontroller hardware, which is the primary
objective of the class.

To complete labs, students use an ESP32, a powerful microcontroller made popular by
the home automation community, programmed with C++, an industry standard in embedded
applications. Since the class does not have any circuit prerequisites and does not 
teach, nor have the time to teach, circuit theory, the class instructors have created
PCBs that are plug-and-play for students to use. These custom PCBs are designed to have
the ESP32 as well as several other electronic components plug directly into it to
give them the ability to work together. Additional components include a small LCD 
touchscreen, buttons, LEDs, and buzzers. 

The labs in ME 400 take an iterative approach where subsequent labs will add features
and functionality on top of the previous week's lab assignment. This gives students
the experience of completing a relatively complex project without overwhelming them 
from the start. 

ME 400 serves as the primary introduction to programming in the mechanical department.
It is also the only class that spends significant teaching about programming and the
only class that gives instructions on microcontrollers. Unfortunately, the class comes
late in the curriculum, after many classes that would benefit from using programming,
and focuses the majority of its time on embedded programming, often to the detriment
of a fundamental understanding of programming. This is a topic of much debate, and
will be addressed in the concluding chapters.

\section{Project Redesign}

Since ME 400 is a programming course, no new assignments are being created for this
project. Instead, a pre-existing lab has been translated from C++ on the ESP32 to
MicroPython on the Raspberry Pi Pico. The translated lab exercise is the concluding
assignment in a string of lab exercises dedicated to programming the game "Simon Says."

The goal of the exercise is to create a replayable, memory-based game that imitates the
children's game Simon Says. To do this, students have to use the LCD screen, five 
buttons, four LEDs, a buzzer, and an IR remote. The lab exercise chosen aims to 
demonstrate how the Pico and MicroPython can handle the same projects and electronics 
that the ESP32 and C++ can. 

To use the LCD screen and the IR remote, drivers needed to be created for the Pico. 
The drivers provided in the repository are modified versions of
existing drivers from https://github.com/rdagger/micropython-ili9341 and 
https://github.com/peterhinch/micropython\_ir. The modified drivers would be given to 
students and no modifications would need to be made to the drivers. 

As previously mentioned, the class uses several custom PCBs to facilitate the use
of different circuit componenets. However, no circuit board was designed for this 
project, so the wirings were all done by hand. Only small changes to hole locations
and traces would be needed to adapt the current PCBs into a useable form for the Pico.

Since no new assignments are being added to the class, the learning objectives for the 
class do not change.

\section{Project Deliverables}

In the GitHub repository associated with this paper, which cna be found in Appendix
\ref{appendix:appendix_github}, the folder titled ``computer-applications-in-me'' 
contains the rewritten lab assignment, code skeleton, and solution for the exercise
mentioned above. The folder also contains a README that details what is in each file
and what software is needed to complete the assignments. Installation instructions can
be found in the ``usage-and-installation'' folder in the GitHub repository.

The integration of these curriculum changes would not be a simple task. Since the 
entire class is structured around C++ and the ESP32, every lecture, assignment, quiz,
and test would have to be translated. In addition to this, new PCBs would need to be 
designed for use with the Pico. 
