\section{Course Overview}

ME 570: Control of Mechanical Systems is a mixed class with two lectures and one lab every week. The lectures focus on teaching
the principles and theory of controls while the lab focuses on designing and testing control systems. Most labs are dedicated to
using and designing PID controllers using different methods, such as frequency analysis or pole anaylsis.

To complete the labs, the class makes use of a custom motor rig called the motorlab. To go along with the motorlab, a graphical 
user interface for controlling the motorlab was also created by Dale Schinstock. This GUI runs as an appliation through MATLAB 
and allows the students to both send commands to the motor as well as collect runtime data from the motor.

Controls also relies heavily on MATLAB for the completion of labs and homeworks. All work is done using .mlx files through a 
licensed copy of MATLAB for a more interactive working environment. To access the controls related commands, a package called 
the Controls Toolbox has to be purchased from MATLAB. 

While Controls uses programming more than any class besides ME 400, little to no instruction is provided on how to program in MATLAB.
The first lab is dedicated to taking the MATLAB Onramp starter course, which takes about 1 to 2 hours to complete. However, the 
Onramp does not address most of the work done in Controls, such as plotting and transfer functions. The Onramp does teach students
the basics of MATLAB syntax, but most students come out of the Onramp just as confused by MATLAB as they were going in.

This disconnect between students and MATLAB becomes a barrier that prevents students from understanding controls. Many students
spend more time trying to understand MATLAB than they do learning controls. 

\section{Lab Assignment Redesigns}

Since ME 570 already fully utlilizes programming in the course, no new assignments are being created. Instead, three pre-existing
labs have been translated from MATLAB to Python. While MATLAB is the industry standard when it comes to control systems, the
license is expensive, students do not have a solid understanding of it, and we do not make use of its most powerful feature,
Simulink. In addition to this, Python has a library, aptly named "control," which has a MATLAB module that imitates the Control
Toolbox, both in form and functionality, from MATLAB. In combination with Jupyter Notebooks and a few custom plotting modifications,
using Python will give a very similar expereince to MATLAB, and will give students that pursue the field of controls any easy
transition to MATLAB. 

All three translated labs make use of the motorlab, which has had its GUI turned into an application, courtesy of the MATLAB
Compiler. This allows the application to be run from an app icon or from the terminal. These labs also make use of a custom plot
command. The plot command highjacks the standard matplotlib plot command and adds the ability to create data tips. The
direction of the datatip changes based on which mouse click is used, and a double click in the plot window will remove all datatips.
More custom functions could be added later to imitate MATLAB functionality as needed, but this was the only one required for
completing these labs.

The first translated lab, Lab 4, is a standard assignment that makes use of the motorlab, reading from the csv file output from the
motorlab, and then plotting the results. The second lab, Lab 10, builds on Lab 4 by adding root locus plots and sisotool. While the
sisotool command in MATLAB is not as powerful as the full designer in MATLAB, it does allow for interactive plots where poles can
be moved and the graphs automatically update. The third lab, Lab 13, focuses on frequency response design methods by making use
of Bode plots.

The labs chosen aim to showcase how the major features used in MATLAB can be emulated in Python. Some features, like Simulink, have
no direct correlation. However, Simulink is only used in one lab, and students only need to make two small changes. Python also
requires more library imports, but these can all be handled by the skeleton and do not pose much concern.

Since no new assignments are being added to the class, the learning objectives 

\section{Project Deliverables}

In the GitHub repository associated with this paper, which can be found in Appendix \ref{appendix:appendix_github},
the folder titled ``control-of-mechanical-systems'' contains both the lab assignments, code skeletons, and solutions for the three
lab assignments explained above. The folder also contains a README that details what is in each file and what software is needed 
to complete the assignments. 

The act of integrating these labs into the class would not be as simple as the integrations for other classes. Since every lab 
uses MATLAB, the other 11 labs would also need to be translated. In addition to this, every computer in the lab would need to 
get the correct applications, extensions, and motorlabGUI executable installed. Instructions for installing everything needed 
can be found in the ``usage-and-installation'' folder in the GitHub repository.
