\section{Course Overview}

ME 533: Machine Design is a lecture based class that focuses on stress/strain analysis, load determination,
and failure theories. Despite being the first machine design class in the mechanical engineering curriculum, 
the material is a continuation of topics covered in two required civil engineering courses: CE 333: Statics
and CE 533: Mechanics of Materials.

As a higher level course that relies on well established theories and analysis methods, machine design
presents several opportunities for programmatic solutions to problems, though Excel calculators may
be more appropriate in some cases. Despite this, machine design does not require programming for 
assignments or projects, though this fact has varied depending on the current instructor. 

Homework assignments typically involve resolving loads on various members, analyzing the internal stresses, and 
determining the safety of a design using the appropriate failure theories. When appropriate, deformation
plots are drawn to highlight points of failure. The class finishes with discussions on spring and fastener
design, which involves heavy iteration.

\section{New Tool for Solving Assignments}

Since no assignments currently make use of programming, a new assignment with two questions adapted from 
\textit{Mechanics of Materials, 10th Edition} \cite{mechanics-of-materials} have been solved 
to demonstrate potential uses for programming in the course. Rather than writing a Python script to solve 
iterative design or strictly numerical problems (see chapters on Thermodynamics, Fluid Mechanics, or 
Heat Transfer for examples of this), Assignment \ref{mach_des_assignment_1} focuses on finding and plotting 
beam deflection using singularity functions. 

\assignments{ME 533: Assignment 1}
\label{mach_des_assignment_1}

\begin{tcolorbox}[breakable, enhanced jigsaw, title=ME 533: Assignment \ref{mach_des_assignment_1}, 
    colframe=ksu-purple, colback=ksu-gray]

    \textbf{Problem Statement}
    \parindent15pt

    The following beam is subjected to the load shown. Determine the equation for deformation. 
    The beam is made of aluminum and has an I = 156 in\textsuperscript{4}. Assume EI is constant. Graph 
    the shear, moment, slope, and deflection.

    \begin{center}
        \includegraphics[width=\textwidth]{mach-des-assingment1.png}
    \end{center}

    \tcblower
    \textbf{Problem Solution}
    \parindent15pt

    For the full solution, see Appendix \ref{appendix:appendix_github}. The following code shows how to
    launch the Singularity Solver from a Jupyter Notebook.

\begin{python}
import subprocess

output = subprocess.run(
    ["python", "-m", "singularity"], 
    shell=True, 
    capture_output=True,
    cwd="./MNE/machine_design"
)
print(output.stdout.decode("ascii"))
\end{python}

After inputting the beam attributes, loading state, and boundary conditions, the following graph
is produced.

\begin{center}
    \includegraphics[width=\textwidth]{mach-des-graph.png}
\end{center}
\end{tcolorbox}

\begin{figure}[h]
    \includegraphics[width=\textwidth]{singularity-gui.png}
    \centering
    \caption{Singularity Solver GUI}
    \centering
    \label{fig:singularity-gui}
\end{figure}

The most simple adoption of programming into singularity function assignments would be requiring computer
generated plots. Students would first solve for reaction forces, create a singularity equation, integrate
several times, and then create a plot using the deformation equation. The solution presented here takes a 
higher level approach by accepting the loading forces on the bar and then creating and integrating the 
singularity equation automatically. This is done through a graphical user interface developed and executed
in Python.

The GUI can be launched in Python (Assignment \ref{mach_des_assignment_1}) or through
the terminal as a Python module. Once launched, the application will open in a new window, which is shown
in Figure \ref{fig:singularity-gui}. The home page contains instructions for using the solver, so the details 
here will be sparing. The second page is dedicated to beam configuration, including length, cross-section, 
material, and boundary conditions. The next page presents various forms of loading to add
to the beam. The last page allows custom materials to be added to the list of materials.

While the GUI currently automates significant portions of singularity problems, several features should be
added at a later date, including non-rectangular cross-sections (though a workaround is presented) and
the ability to add supports to the beam. This would give the program the ability to calculate the reaction
forces and determine boundary conditions without relying on the user to correctly find and input them.

Since this application takes most of the work out of the hands of the students, it would not be a good tool
for teaching singularity functions. Rather, the application would be best utilized when finding the shear,
moment, slope, or deformation of a beam is one part in the context of a larger problem. Additionally, 
it would serve as a realistic look at how most analysis is done in industry - by a computer.

Using programming in this manner gives students a better idea of how real problems are solved by
introducing them to a more efficient and powerful solution method. These questions would also 
directly correlate to Abet Student Outcomes 1, 6, and 7, as seen in Appendix 
\ref{appendix:appendix_abet} and weekly correlates to Outcome 3.

\section{Recommendations for Course Integration}

Show in class example, really used as a tool in the process of analyzing mechanical problem, need to apply
graphs should be a required portion of the assignments, providing motivation for it

\section{Project Deliverables}

In the GitHub repository associated with this paper, which can be found in Appendix \ref{appendix:appendix_github},
the folder titled ``8-machine-design'' contains both the problem statements and solution guides for the two questions
introduced in the previous section. The folder also contains a README that details what is in each file and 
what software is needed to complete the assignments. 

For these projects to be added to the class, the instructor would  need to give the skeleton files to 
students as a problem statement and the MNE folder as Python library. Using one of the two questions as an in-class
example would serve as a good opportunity to remind students how to use Python with Jupyter Notebooks and to 
demonstrate how to open and use the singularity function GUI.
