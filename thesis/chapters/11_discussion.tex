\section{Issues and Concerns}

While the previous chapters proved that Python can be effectively
used to solve problems in the mechanical engineering field, it did 
little to address the issues that would come with the adoption 
of a unified programming ecosystem, both as it relates to the
concept itself as well as the limitations of the proposed solution.

\subsection{Version Control}

While open source software is often viewed as desirable
thanks to the no-cost price tag and active development that it 
promotes, it comes with a downside often avoided by premium
software: version control. Python releases a new minor version
every year and regularly releases patches for supported versions.
On top of that, several of the libraries and extensions 
used in this project are written and controlled by various 
groups and individuals that make no guarantee of longevity or future
support. This leaves a substantial burden on the department to
moderate and maintain a locally hosted version of Python and the
libraries required by each class, as well as Visual Studio Code and
its extensions. As the number of classes utilizing programming
increases, so does the difficulty of validating a functioning stack.

With that said, the majority of the software being used is well
maintained and rarely makes breaking changes. Libraries like $numpy$,
$matplotlib$, and $scipy$ are staples of the Python ecosystem. Libraries
like $mendeleev$ or $pyXSteam$ that rarely see updates and rely on older
versions of more popular libraries present a larger concern.

Likewise, updates to VSCode and its extensions often
amount to bug fixes or small feature changes and pose little threat
to the functionality of the development environment as it relates to
the mechanical department.

\subsection{Industry Standards}

By adopting Python as the only language taught in core curriculum
classes, students would lose exposure to C/C++, the standard for 
embedded software design, and MATLAB, the standard for control system
design. While these languages would hopefully still be taught
through electives such as ME 615: Applications in Mechatronics 
and ME 640: Control of Mechanical Systems II, the general student body
may never learn either language. 

Both C/C++ and MATLAB are deeply rooted as the standards in their fields,
and for good reason. C/C++ is a compiled language that is designed
to access lower levels of hardware than Python (though MicroPython does
enable low-level access, it is still an interpreted language and can
never match the speed of compiled code), and MATLAB has decades of controls
development in the form of Simulink, the true standard for control system 
design.

With that being said, Python is the industry standard for data analysis 
and numerical methods, and these are the applications that many mechanical
engineers are interested in. While plenty of mechanical engineers work 
with embedded devices or design control systems, job listings and hiring managers are
transitioning more towards recruiting more in the Electrical and Computer Engineering field.
The typical mechanical engineers will be interested in analyzing data produced 
by a machine or with running preliminary calculations before committing
the time and resources to a lengthy simulation or prototype. These are 
the applications that will see the most benefit from learning Python.

\subsection{Opportunity Cost}

With the addition of programming to a class, most classes would need to devote
at least a portion of one lecture to showcasing the libraries needed to 
solve homework problems. For classes before ME 400, an entire lecture may
be necessary. This would fit best alongside a typical by-hand, in-class
example to compare the two methods.

The addition of programming would also lead to additional questions for
instructors regarding the setup and installation of the environment. While
the setup steps are relatively easy compared to other languages, it will
still be unfamiliar to students and require teachers to be knowledgeable
on the subject.

\section{Recommendations}

I recommend classes that make significant use of any property tables, 
iterative design, or graphing add at least a single assignment that 
requires the use of Python and the relevant library to complete an assignment
typical of the subject field. As it relates to this project, that would include
DEN 161: Engineering Problem Solving, ME 513: Thermodynamics, NE 495: Elements 
of Nuclear Engineering, ME 571: Fluid Mechanics, ME 533: Machine Design, and 
ME 573: Heat Transfer. The addition of programming to these classes will 
increase students' abilities to efficiently solve problems with little downside. 

Given that this paper was not afforded the time to implement these changes
and observe the change in students, further recommendations are hinged on the
result of the first recommendation.

Should students continue to struggle with programming, especially in
classes prior to ME 400, I recommend that a programming class either be 
added or moved into the first four semesters of the degree, preferably 
the first three. The nature of this change could take any of several forms. 

The easiest change would involve moving ME 400 forward two semesters in the
flowchart and moving CHE 354/355 and IMSE 250 back two semesters. Currently,
ME 400 does not utilize information from MATH 340 and neither CHE 354/355 nor
IMSE 250 are prerequisites for other classes, making this change low friction. 

Another possible change is the addition of a programming class, like CIS 209: 
Computer Programming for Engineers, into the curriculum. In addition to 
giving students a better foundation in programming through the introduction
of an extra class, this change would also free up time in ME 400. In previous
years, CIS 209 was a prerequisite to taking ME 400. However, as the number
of required credit hours dropped, CIS 209 was removed from the curriculum,
but ME 400 did not adjust enough to compensate for this course change.

Finally, a more drastic change would be a redesign of the ME 400 and 
ECE 519: Circuits and Controls/ME 519: Circuits for MNE classes. Rather
than acting as a microcontroller class, ME 400 can transition into a dedicated
programming class that spends time teaching the basics of programming, how
to use external libraries, and how to solve basic problems using Jupyter
Notebooks. Then, ECE/ME 519 could be transitioned to a circuit and 
microcontroller class that lectures on circuit theory and the fundamentals
of microcontrollers while utilizing labs that give practical experience building
circuits and writing embedded code. Alternatively, ME 519 could be replaced 
with the existing ECE 241: Introduction to Computer Engineering which covers 
these exact topics. The scope of this class would reduce
the amount of time spent on both circuit theory and microcontrollers, which
may not be an acceptable trade-off. With that being said, the majority
of the information taught in ME 519 is either repeated in ME 535:
Measurements and Instrumentation or is not used again in the curriculum.

If, however, the students continue to struggle after ME 400, I would 
recommend converting ME 400 and ME 570 into Python-based classes to give
students a more consistent approach to programming as well as moving 
ME 400 earlier in the curriculum to solidify the foundations before 
requiring its use in other mechanical classes.
