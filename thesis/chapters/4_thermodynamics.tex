\section{Course Overview}

ME 513: Thermodynamics is a lecture based class that focuses on the interplay of heat, work, and energy in both
open and closed loop systems. While there is no lab portion to the class, significant lecture time is spent
solving problems and working examples, and a new homework is assigned every lecture to give students a chance
to apply the new concepts taught that day.

Most problems in thermodynamics follow a similar solution path. First, students must recognize the different
states in the system. Once the different states are identified, the values for temperature, power, energy,
entropy, etc. must be found for each state. The exact values needed, and retrievable, vary depending on the
given system. This process is known as setting the state. Once the state has been set, known laws of
thermodynamics can be applied and the question can be solved.

For most questions, the majority of time is spent setting the state. This involves using known properties, 
usually given in the problem statement, to look up values in property tables in a thermodynamics textbook. 
This often proves to be a very time consuming process, especially when interpolation is required to get 
property values. And once iteration is introduced, such as in a design problem, manually searching through
tables becomes impractical, if not impossible. This is where the use of a programming language would become 
highly beneficial.

Thermodynamics, as it currently stands, make no use of programming to complete assignments.
This can be attributed to ME 513 coming prior to ME 400, the only programming class in the 
curriculum. However, with the recent programming additions made to DEN 161, students have been introduced
to programming and, with the help of an example, should be capable of solving these questions as a small
project or assignment.

\section{New Assignments}

Since no assignments currently make use of programming, two new asssignments have been created to demonstrate
how programming could be beneficial to the course. Both assignments will make use of a Python library called
PYroMat. This is a free, open source library dedicated to making thermodynamic properties readily available.
They will also both make use of Jupyter Notebooks for cleanly presenting questions, commentaries, and code
while solving the question.

The first assignment, Question 8.23, is a five state system that does not require iteration or plotting. It 
follows the structure of a typical question in thermodynamics well. The student must identify the states, list
the known properties for each state, and then either search for additional values in a table or calculate them 
with known values. After this, the laws of thermodynamics are applied, equations are balanced, and the question
is solved.

The second assignment, Question 8.33, better showcases the benefits of programming. This question uses both
iteration and plotting to show the changes in quality and thermal efficiency as the pressure increases. If 
done by hand, this would be a difficult and tedious task. But when programming, it only requires a few extra
lines of code.

Using programming in this manner gives students a better idea of how real problems are solved by introducing
them to a more efficient and powerful solution method. These questions would also directly correlate to Abet 
Student Outcomes 1, 6, and 7 and weakly correlate to Outcome 3, as seen in Appendix \ref{appendix:appendix_abet}.

\section{Project Deliverables}

In the GitHub repository associated with this paper, which can be found in Appendix \ref{appendix:appendix_github},
the folder titled ``thermodynamics'' contains both the problem statements and solution guides for the two questions
introduced in the previous section. The folder also contains a README that details what is in each file and 
what software is needed to complete the assignments. 

For these projects to be added to the class, the instructor would simply need to give the skeleton files to 
students as a problem statement. It may be beneficial to use question 8.23 as an in-class example both to serve 
as a reminder of how to use Python with Jupyter Notebooks as well as a demonstration of how to use PYroMat to 
access material properties and change property units.