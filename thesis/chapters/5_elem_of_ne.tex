\section{Course Overview}

NE 495: Elements of Nuclear Engineering is a lecture based class that focuses on introducing
the fundamental concepts of chemistry and physics that form the field of nuclear engineering 
to mechanical engineering students. The class also serves as the sole entry point into the nuclear
program at K-State. 

As class with no lab section, homework assignments and tests make up the majority of the grade
for the class. Currently, the class also has one project where students use a Geiger counter 
developed for the class to measure ionizing radiation. 

As is the case with ME 513: Thermodynamics, this class comes before ME 400 and, therefore, does
not utilize programming for any homework assignments or projects. With the changes to DEN 161,
and with the addition of programming to other classes, such as ME 513, students should be able 
to solve these problems.

\section{New Assignments}

While no assignments currently make use of programming, many questions lend themselves nicely
to being solved with programming. This is thanks to the heavy reliance on tables and
constants used in solving NE problems. To demonstrate, three questions, one from an exam
and two from a homework, have been solved using Python with the help of a few open source
libraries: mendeleev \cite{Mentel_mendeleev_-_A_2021}, physdata \cite{physdata2016}, and scipy
\cite{2020SciPy-NMeth}.

Unfortunately, no single library contains the full collection of properties to solve the range
of questions presented in NE 495. While inconvenient, this issue serves to highlight the
benefits of making an MNE library that can wraps around the necessary packages

The first assignment, Assignment \ref{nuclear_assignment_1}, taken from the first exam in the class, 
is a Q-value question that uses 
the mendeleev library to pull element properties. The library contains information on the elements
and their isotopes that, once familiar, provide quick access to the mass and abundance values
needed to solve Q-value problems. Unfortunately, the library outputs a list of isotopes and 
has no direct method for retrieving a specific mass number.

\assignments{NE 495: Assignment 1}
\label{nuclear_assignment_1}

\begin{tcolorbox}[breakable, enhanced jigsaw, title=NE 495: Assignment \ref{nuclear_assignment_1}, 
    colframe=ksu-purple, colback=ksu-gray]

    \textbf{Problem Statement}
    \parindent15pt

    It has been proposed to extract uranium from seawater to produce nuclear reactor fuel. 
    Assume that the total volume of the oceans is 1.3x10\textsuperscript{9} km\textsuperscript{3}, the uranium 
    concentration in the ocean is 3 parts per billion water molecules, and the total annual 
    electricity consumption in the U.S. is 4,095 Billion kWh (1 kWh = 3.6x10\textsuperscript{6} J). Assume further 
    that every single 235U nucleus fissions according to the reaction n + 235U → 139Ba + 95Kr + 2 n 
    and that 30\% of all the energy can be recovered for electrical production. How many years could the oceanic 
    uranium power the U.S.?

    Be sure to specify all assumptions, provide commentary on the numbers being calculated, and comment on the 
    functionality of the code. The necessary imports are included below.

    \tcblower
    \textbf{Problem Solution}
    \parindent15pt

    For the full solution, see Appendix \ref{appendix:appendix_github}. Below is an example of 
    importing and using the mendeleev library to get atomic weights.

\begin{python}
import mendeleev as md

H = md.element("Hydrogen")
O = md.element("Oxygen")

water_molecule_mass = 2*H.atomic_weight+O.atomic_weight
\end{python}

Other physical constants can be found in scipy's constants library.

\begin{python}
from scipy.constants import physical_constants

# 931.5 MeV / amu
c_2 = physical_constants[
    "atomic mass constant energy equivalent in MeV"][0]

# 1.008 amu
neutron_mass=physical_constants["neutron mass in u"][0]
\end{python}

\end{tcolorbox}

The second and third questions, both in Assignment \ref{nuclear_assignment_2}, are taken 
from Homework 20 and deal with photon interactions. 
These questions make use of the physdata library to get attenuation coefficients for different
molecules (water, in this example). Similar to mendeleev, this library does not have an easy, 
built-in method for getting this coefficient values at non-standard energy values. To get around
this, students are encouraged to use numpy.interp to find the desired value.

\assignments{NE 495: Assignment 2}
\label{nuclear_assignment_2}

\begin{tcolorbox}[breakable, enhanced jigsaw, title=NE 495: Assignment \ref{nuclear_assignment_2}, 
    colframe=ksu-purple, colback=ksu-gray]

    \textbf{Problem Statement}
    \parindent15pt

    What fraction (not percent) of 2.5 MeV photons interact within 1 foot of water? What fraction 
    (not percent) of 2.5 MeV photons are absorbed by 1 foot of water?

    \tcblower
    \textbf{Problem Solution}
    \parindent15pt
    
    For the full solution, see Appendix \ref{appendix:appendix_github}. Below is a walkthrough
    of the solution to the first question.

    First, get the table of attenuation coefficients using physdata. The list is converted to 
    a numpy array for easier handling in the next step.

\begin{python}
# cm-1
water_table = np.array(
    xray.fetch_coefficients("water", 1))
\end{python}

Next, interpolate using `numpy`'s `interp` function to get the correct coefficient. Here column 
0 is the energy levels and column 1 is the interaction coefficient.

\begin{python}
mew = np.interp(2.5, water_table[:,0], water_table[:,1])
\end{python}

Using the equation $ \frac{I}{I_0} = exp(-\mu x) $, we can get the ratio that does not 
interact with the water. To get the ratio that does interact with the water, take 
$ 1 - l_{ratio} $.

\begin{python}
x = 12 * 2.54 # 12 inches * 2.54 cm/inch
I_ratio = np.exp(-mew * x)
interact_ratio = 1 - I_ratio
print(f"I/I0 = {interact_ratio:.3f}")
\end{python}

\end{tcolorbox}

Using programming in this manner gives students a better idea of how real problems are solved by
introducing them to a more efficient and powerful solution method. These questions would also 
directly correlate to ABET Student Outcomes 1, 6, and 7, as seen in Appendix 
\ref{appendix:appendix_abet}. With the addition of a question that utilizes basic plotting, a 
weak correlation to Outcome 3 could also be added.

\section{Recommendations for Course Integration}

After completing the programming assignments in NE 495, students should be able to recognize when programming
techniques can be applied to solve a problem, explain how tools are used, and implement a program to solve 
a nuclear engineering question. Of these objectives, an emphasis should be placed on the students' ability to apply
tools given by the instructor.

To accomplish this, NE 495 should adopt at least two required programming questions and add the option for 
students to continue using programming to solve homework questions. For example, the first assignment could
consist of a traditional by-hand solution with the second question being the same problem but with an additional
requirement of solving through programming. The second assignment could be a small project or a dedicated
programming assignment that requires graphing, iteration, or numerous table values. 

To supplement these assignments, at least one lecture should be dedicated to showing how to solve a nuclear 
engineering question with Python. In addition, a ``cheat sheet'' of commands, library calls, and other 
necessary operations should be given to students. By providing a sheet of code commands, students can focus 
on understanding and applying information.

\section{Project Deliverables}

In the GitHub repository associated with this paper, which can be found in 
Appendix \ref{appendix:appendix_github}, the folder titled ``5-elements-of-nuclear-engineering''
contains both the problem statements and solution guides for the three questions introduced in 
the previous section. The folder also contains a README that details what is in each file and 
what software is needed to complete the assignments. 

For these projects to be added to the class, the instructor would simply need to give the 
skeleton files to students as a problem statement. It may be beneficial to use one of the questions 
as an in-class example both to serve as a reminder of how to use Python with Jupyter Notebooks 
as well as a demonstration of how to use the data libraries to access material properties.