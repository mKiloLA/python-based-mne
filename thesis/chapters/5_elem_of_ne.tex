\section{Course Overview}

NE 495: Elements of Nuclear Engineering is a lecture based class that focuses on introducing
the fundamental concepts of chemistry and physics that form the field of nuclear engineering 
to mechanical engineering students. The class also serves as the sole entry point into the nuclear
program at K-State. 

As class with no lab section, homework assignments and tests make up the majority of the grade
for the class. Currently, the class also has one project where students use a geiger counter 
developed for the class to measure ionizing radiation. 

As is the case with ME 513: Thermodynamics, this class comes before ME 400 and, therefore, does
not utilize programming for any homework assignments or projects. With the changes to DEN 161,
and especially should changes come to other classes, such as ME 513, students should be able 
to solve these problems.

\section{New Assignments}

While no assignments currently make use of programming, many questions lend themselves nicely
to being solved with programming. This is thanks to the heavy reliance on tables and
constants used in solving NE problems. To demonstrate, three questions, one from an exam
and two from a homework, have been solved using Python with the help of a few open source
libraries: mendeleev, physdata, and scipy.

Unfortunately, no single library contains the full collection of properties to solve the range
of questions presented in NE 495. While inconvenient, this issue serves to highlight the 
importance of creating a package for the MNE department that consolidates and standardizes the
information and structure of property data. 

The first question, taken from the first exam in the class, is a Q-value question that uses 
the mendeleev library to pull element properties. The library contains information on the elements
and their isotopes that, once familiar, provide quick access to the mass and abundance values
needed to solve Q-value problems. Unfortunately, the library outputs a list of isotopes and 
has no method for retrieving a specific mass number. Because of this, a wrapper function was
written to allow for easy access to individual isotopes. This function mimics the work that
would be done in an MNE library.

The second and third questions are both taken from homework 20 and deal with photon interactions. 
These questions make use of the physdata library to get attenuation coefficients for different
molecules (water, in this example). Similar to mendeleev, this library does not have an easy, 
built-in method for getting this coefficient values at non-standard energy values. Once again, a
wrapper function was written to allow for easy access to interpolated values.

Using programming in this manner gives students a better idea of how real problems are solved by
introducing them to a more efficient and powerful solution method. These questions would also 
directly correlate to Abet Student Outcomes 1, 6, and 7, as seen in Appendix 
\ref{appendix:appendix_abet}. With the addition of a question that utilizes basic plotting, a 
weak correlation to Outcome 3 could also be added.

\section{Project Deliverables}

In the GitHub repository associated with this paper, which can be found in 
Appendix \ref{appendix:appendix_github}, the folder titled ``elements-of-nuclear-engineering''
contains both the problem statements and solution guides for the three questions introduced in 
the previous section. The folder also contains a README that details what is in each file and 
what software is needed to complete the assignments. 

For these projects to be added to the class, the instructor would simply need to give the 
skeleton files to students as a problem statement. It may be beneficial to use one of the questions 
as an in-class example both to serve as a reminder of how to use Python with Jupyter Notebooks 
as well as a demonstration of how to use the data libraries to access material properties.