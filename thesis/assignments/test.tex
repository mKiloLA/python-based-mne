\textbf{Problem Statement}

Water is the working fluid in a Rankine cycle. Steam exits the steam generator at 1500 
lbf/in\textsuperscript{2}  and 1100°F. Due to heat transfer and frictional effects in 
the line connecting the steam generator and turbine, the pressure and temperature at the 
turbine inlet are reduced to 1400 lbf/in\textsuperscript{2} and 1000°F, respectively. Both the 
turbine and pump have isentropic efficiencies of 85\%. Pressure at the condenser inlet 
is 2 lbf/in\textsuperscript{2}, but due to frictional effects the condensate exits the 
condenser at a pressure of 1.5 lbf/in\textsuperscript{2} and a temperature of 110°F. The 
condensate is pumped to 1600 lbf/in\textsuperscript{2} before entering the steam generator. 
The net power output of the cycle is 1x10\textsuperscript{9} Btu/h. Cooling water experiences 
a temperature increase from 60°F to 76°F, with negligible pressure drop, as it passes 
through the condenser. Determine for the cycle:

\begin{enumerate}
    \item the mass flow rate of steam, in lb/h.
    \item the rate of heat transfer, in Btu/h, to the working fluid passing through the steam 
    generator. 
    \item the thermal efficiency.
    \item the mass flow rate of cooling water, in lb/h.
\end{enumerate}

Be sure to leave specify all assumptions and comment on the functionality of the code. 
To access thermodynamic tables, install PYroMat using the cell below.

\tcblower

\begin{python}
    import pyromat

    pyromat.config["unit_energy"] = "BTU"
    pyromat.config["unit_force"] = "lbf"
    pyromat.config["unit_length"] = "in"
    pyromat.config["unit_mass"] = "lb"
    pyromat.config["unit_matter"] = "lb"
    pyromat.config["unit_molar"] = "lbmol"
    pyromat.config["unit_pressure"] = "psi"
    pyromat.config["unit_temperature"] = "F"
    pyromat.config["unit_volume"] = "ft3"
    H2O = pyromat.get("mp.H2O")
\end{python}
