Full project documentation can be found at the following GitHub repository: 

\begin{center}
    \url{https://github.com/mKiloLA/python-based-mne/tree/develop}
\end{center}

\subsection*{Implementing a Cohesive Programming Ecosystem in Mechanical Engineering}

This repository serves as the project base for a thesis entitled
\textit{Implementing a Cohesive Programming Ecosystem in Mechanical Engineering}. 
Of the ten folders, eight are dedicated to different courses in the Mechanical 
Engineering curriculum at Kansas State University. Each class folder contains the 
resources needed to give and evaluate various engineering problems related to the 
subject. As the title suggests, the primary goal of the project is to promote the 
use of a single programming language and environment to give students a stronger 
foundation in programming. As such, each assignment is solved using Python in Visual 
Studio Code to give a consistent learning environment. For a more detailed look into 
the reasoning for the project and background on the classes, see the associated 
\href{https://github.com/mKiloLA/python-based-mne/tree/develop/thesis}{thesis}.

\subsection*{Folders in the Repository}

The follow directory only lists the top level folders. For a breakdown of each project, please see the README files in each project folder.
\begin{itemize}
    \item \href{https://github.com/mKiloLA/python-based-mne/tree/develop/3-engineering-problem-solving}{DEN 161: Engineering Problem Solving}: Assignment information for an Intro to Engineering class.
    \item \href{https://github.com/mKiloLA/python-based-mne/tree/develop/4-thermodynamics}{ME 513: Thermodynamics}: Assignment information for a Thermodynamics class.
    \item \href{https://github.com/mKiloLA/python-based-mne/tree/develop/5-elements-of-nuclear-engineering}{NE 495: Elements of Nuclear Engineering}: Assignment information for a Nuclear Engineering class.
    \item \href{https://github.com/mKiloLA/python-based-mne/tree/develop/6-computer-applications-in-me}{ME 400: Computer Applications in Mechanical Engineering}: Assignment information for a Computer Applications class.
    \item \href{https://github.com/mKiloLA/python-based-mne/tree/develop/7-fluid-mechanics}{ME 571: Fluid Mechanics}: Assignment information for a Fluid Mechanics class.
    \item \href{https://github.com/mKiloLA/python-based-mne/tree/develop/8-machine-design}{ME 533: Machine Design}: Assignment information for a Machine Design class.
    \item \href{https://github.com/mKiloLA/python-based-mne/tree/develop/9-control-of-mechanical-systems}{ME 570: Control of Mechanical Systems}: Assignment information for a Control Systems class.
    \item \href{https://github.com/mKiloLA/python-based-mne/tree/develop/10-heat-transfer}{ME 573: Heat Transfer}: Assignment information for a Heat Transfer class.
    \item \href{https://github.com/mKiloLA/python-based-mne/tree/develop/thesis}{Thesis}: Source files for the Thesis associated with this repository.
    \item \href{https://github.com/mKiloLA/python-based-mne/tree/develop/usage-and-installation}{Usage and Installation}: Instructions for installing and setting up the environment needed to complete the projects in this repository.
\end{itemize}

\subsection*{Usage and Installation}

Since this repository a represents a collection of projects, the required software 
installation and usage varies by class. Each class has a `README.md' file that details 
the software, Visual Studio Code extensions, and Python packages required to run complete 
the assignments. In general, each project requires a download of Python, Visual Studio 
Code, and a handful of extensions for Visual Studio Code. The 
\href{https://github.com/mKiloLA/python-based-mne/tree/develop/usage-and-installation}{usage-and-installation} 
contains guides for installing Python, Visual Studio Code, and the extensions. 
The Python installation guide also contains information on how to install the Python 
packages used across the entire project. Alternatively, the dependencies can be 
installed from the requirements.txt file by running the follow command from the root 
directory of `usage-and-installation':

\begin{center}
    \pyth{pip install -r requirements.txt}
\end{center}
